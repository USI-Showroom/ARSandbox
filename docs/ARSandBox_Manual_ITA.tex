\documentclass[12pt]{article}
\usepackage[margin=2cm]{geometry} 
\usepackage{titling}
\usepackage{graphicx}
\usepackage{float}
\usepackage[hidelinks]{hyperref}
\usepackage[italian]{babel}
\usepackage{subcaption}

\setlength\parindent{0pt}
\setlength{\parskip}{1em}
\setlength{\droptitle}{-2cm}

\title{Istruzioni d'uso AR Sandbox (ad uso interno)}
\author{Università della Svizzera italiana}
\date{Versione \today}


\begin{document}
\maketitle
\tableofcontents
\newpage


\section{Installazione}\label{installation}	

	\subsection{Componenti}
	
		\begin{itemize}
			\item PC
			\item Cavo DVI lungo e 2x cavo alimentazione
			\item Proiettore
			\item Kinect v2
			\item Materiale di montaggio Kinect e proiettore
			\item (opzionale) Altoparlanti
		\end{itemize}
		
		
	\subsection{Montaggio}
	
		\begin{enumerate}
			\item Collegare l'alimentazione e il cavo DVI al proiettore (entrata MONO 2D)
			\item Montare il proiettore sull'apposito supporto e far passare i cavi
			nell'apertura
			\item Montare il Kinect sul supporto dedicato e far passare il cavo nell'apertura
			\item Collegare l'alimentazione e il cavo DVI al PC, ed eventualmente gli altoparlanti
			\item Collegare l'alimentazione del Kinect e il cavo USB (blu) ad una porta USB 3.0 (blu) del PC
		\end{enumerate}
		
		
	\subsection{Avvio}

		Accendere il proiettore, dopodiché avviare il PC e cliccare sul collegamento
		\texttt{ARSandBox} sul Desktop. Controllare che il proiettore \textbf{non} sia in modalità 3D
		(menu $\rightarrow$ stereo $\rightarrow$ stereo mode).
		
		
\newpage
\section{Utilizzo}

	\subsection{Tasti utili}
	
	\begin{tabular}{l l}
		ESC & exit\\
		P & save mesh/point cloud\\
		F5 & toggle set-up mode\\
		- & turn off visualization and show only sand\\
	\end{tabular}

	In modalità set-up

	\begin{tabular}{l l}
		Shift & slower movement\\

		F1 & terrain 1\\
		F2 & terrain 2\\
		F3 & terrain 3\\

		1/2/3/4 & select corners\\
		5 & scale\\
		6 & move\\

		W/A/S/D & move the current selection in the four directions\\

		Space & toggle Kinect calibration/projector calibration\\

		9 & save current calibration to disk\\
		0 & load calibration from disk\\

		U/J & increase/decrease the minimum height\\
		I/K & increase/decrease the maximum height\\
	\end{tabular}
		
		
\section{Codice sorgente}

	Il codice sorgente dell'applicazione, assieme alle istruzioni per compilarla, si trova alla pagina \url{https://github.com/USI-Showroom/ARSandbox}.
		
	
\end{document}
